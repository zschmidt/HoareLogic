\documentclass[11pt]{article}
%%% note templates for memorizing some concrete mathematical calculations. 
%%% serve as a complementary to the notes in the One-Note. 
%%% keep a rough time-line record. 

\def\draft{1}  % 1 for draft version (eg including author notes), 0 for non-draft version

%---------Fonts---------%
\usepackage{url}
% \usepackage{palatino}
% \usepackage{mathpazo}
\usepackage{color}
\usepackage{graphicx}
%---------Margins and Length Adjustments---------%
\usepackage[margin=1in]{geometry}
\usepackage{mdwlist}
\setlength{\parskip}{1em}
\setlength{\parindent}{0em}
\newcommand{\tinyspace}{\mspace{1mu}}

%---------Math---------%
\usepackage{amssymb, amsthm, amsmath}
%This package allows for the natural deduction proofs
%Can be found here: http://math.ucsd.edu/~sbuss/ResearchWeb/bussproofs/bussproofs.sty
\usepackage{bussproofs}


%---------Comments-------%
\ifnum\draft=1
\newcommand{\Wnote}[1]{{\bf\color{blue} [Xiaodi's Note: #1]}}
\newcommand{\cmt}[1]{{\color{red} #1}}
\else
\newcommand{\Wnote}[1]{{}}
\newcommand{\cmt}[1]{{#1}}
\fi

%-------- Time ----------%
\newcommand{\notedate}[1]{\hfill \emph{#1}}

\begin{document}
\bibliographystyle{plain}

\def\<{\langle}
\def\>{\rangle}
\def\N{\mathbb{N}}
\def\Nwz{\mathbb{N}_0}
\def\Q{\mathbb{Q}}
\def\C{\mathbb{C}}
\def\R{\mathbb{R}}
\def\eps{\varepsilon}
\newcommand{\fid}{\operatorname{F}}
\newcommand{\h}{\mathcal{H}}
\newcommand{\norm}[1]{\left\lVert\tinyspace#1\tinyspace\right\rVert}
\newtheorem{theorem}{\bf Theorem}[section]
\newtheorem{lem}{Lemma}[section]
\newtheorem{condition}{\bf Condition}[section]
\newtheorem{corollary}{\bf Corollary}[section]
\newtheorem{proposition}{Proposition}
\newtheorem{observation}{Observation}
\newtheorem{conj}{Conjecture}
\newcommand{\bra}[1]{\langle #1|}
\newcommand{\ket}[1]{|#1\rangle}
\newcommand{\braket}[3]{\langle #1|#2|#3\rangle}
%%inner product
\newcommand{\ip}[2]{\langle #1|#2\rangle}
%%outer product
\newcommand{\op}[2]{|#1\rangle \langle #2|}
\newcommand{\symm}[1]{{\mathfrak S}_{#1}}
\newcommand{\unitary}[1]{U_{#1}}
\newcommand{\srep}{{\mathtt P}}
\newcommand{\urep}{{\mathtt Q}}

\newcommand{\tr}{{\rm tr}}
%\newcommand{\mod}{{\rm mod}}
\newcommand {\E } {{\mathcal{E}}}
\newcommand {\F } {{\mathcal{F}}}
\newcommand {\diag } {{\rm diag}}

%---------Metadata---------%
\title{Introduction to Hoare Logic}
\author{ 
}
\date{\today}

%---------Main Document---------%

\maketitle

%---------Abstract---------%
\begin{abstract}
Hoare Logic was created by C.A.R. Hoare in an attempt to put computer programming on the same logical foundations as mathematics. It consists of a set of axioms and rules of inference which can be used to show proofs of the properties of particular computer programs.
\end{abstract}

\section{Introduction}

I'm going to (quickly) talk about the programming language IMP, then we're going to see how Hoare logic makes it easier to derive proofs of programs.

IMP is a \textit{very} basic imperative programming language which can only store and operate on numeric values.


\section{Syntax}
The abstract syntax of IMP is defined inductively as follows:

\begin{description*}

\item[$s ::= $] $\textbf{skip } |\ x := e\ |\ s; s\ |\ \textbf{if } e\ s\ s |\ \textbf{while } e\ s$

\item[$e ::= $] $ c\ |\ x\ |\ e + e\ |\ e * e $

\item[$c \in \mathbb{Z}$]

\item[$x \in $] $\{ x_1, x_2, ..., y_1, y_2, ..., z_1, z_2, ..., ... \}$

\end{description*}

\subsection{Heap}
In order to store our variables somewhere, we need to introduce the concept of a \textit{heap}, which is a total function from variables to constants. More formally, a heap is:

\begin{description*}

\item[$H ::= $] $\cdot \ |\ H, x\to c$

\end{description*}

And the associated lookup function:

\[ 
H(x) = \begin{cases} 
      c & \textbf{if } H = H', x\to c \\
      H'(x) & \textbf{if } H = H', y\to c', \textbf{ and } y\neq x\\
      0 & \textbf{if } H = \cdot 
   \end{cases}
\]


We can now associate \textit{judgements} to our programs. The syntax ``$H; e \Downarrow c$'' is read ``$e$ evaluates to $c$ under heap $H$''.

\section{Semantics}
Now that we have a notion of the symbols that can be used, lets see what happens when we give them meaning!

\begin{description*}

\item[Constant:]

\begin{prooftree}
	\AxiomC{ }
	\UnaryInfC{$H; c \Downarrow c$}
\end{prooftree}

\item[Variable:]


\begin{prooftree}
	\AxiomC{ }
	\UnaryInfC{$H; x \Downarrow H(x)$}
\end{prooftree}


\item[Add:]

\begin{prooftree}
	\AxiomC{$ H; e_1 \Downarrow c_1 $}
	\AxiomC{$ H; e_2 \Downarrow c_2 $}
	\BinaryInfC{$ H; e_1 + e_2 \Downarrow c_1 + c_2 $}
\end{prooftree}

\item[Multiply:]


\begin{prooftree}
	\AxiomC{$ H; e_1 \Downarrow c_1 $}
	\AxiomC{$ H; e_2 \Downarrow c_2 $}
	\BinaryInfC{$ H; e_1 * e_2 \Downarrow c_1 * c_2 $}
\end{prooftree}

\end{description*}


\section{Deductive Logic}

With the above, we have enough to prove (contrived) programs in IMP! For example, if we wanted to show that the program ``$\cdot ,y\to 4 ; (3 + y)+5 \Downarrow 12$'' is correct, we could simply apply deductive logic to see a complete derivation as follows:

\begin{prooftree}
\AxiomC{ }
\UnaryInfC{$\cdot, y\to 4 ; y \Downarrow 4$}
\AxiomC{ }
\UnaryInfC{$\cdot, y\to 4 ; 3 \Downarrow 3$}
\BinaryInfC{$\cdot, y\to 4 ; 3 + y \Downarrow 7$}
			\AxiomC{ }
			\UnaryInfC{$\cdot, y\to 4 ; 5 \Downarrow 5$}
\BinaryInfC{$\cdot ,y\to 4 ; (3 + y)+5 \Downarrow 12$}

\end{prooftree}

\newpage

\section{Hoare Logic}
In 1969, C.A.R. Hoare noted that ``[c]omputer programming is an exact science in that all the properties of a program and all the consequences of executing it can be found out from the text of the program itself by means of purely deductive reasoning.''\cite{hoare1969axiomatic} \\

The central feature of Hoare logic is the Hoare triple. A triple describes how the execution of a piece of code changes the state of the computation. A Hoare triple is of the form:

\vspace{-1.5em}
\begin{align*}
\{P\} C \{R\}
\end{align*}

The above is read \textit{``If the assertion P is true before initiation of a program C, then the assertion R will be true on its completion''}.

As should be fairly evident, this appears (on the surface) to look similar to our IMP syntax -- we have some precondition (the heap in IMP), a program (an expression in IMP), and a resulting postcondition (the resulting heap).

\subsection{Rules}
The original paper lays out the following rules\footnote{The notation has changed since the original paper. The notation in this paper follows modern convention.}:

\begin{description*}

\item[Assignment:]

\begin{prooftree}
	\AxiomC{ }
	\UnaryInfC{$\{P[E/x]\} x := E \{P\}$}
\end{prooftree}

\textit{(where $P[E/x]$ denotes the assertion $P$ in which each free occurrence of $x$ has been replaced by the expression $E$)}

\item[If Statements:]


\begin{prooftree}
	\AxiomC{$\{B \land P\} S \{Q\} $}
	\AxiomC{$\{\lnot B \land P \} T \{Q\}$}
	\BinaryInfC{$\{P\} \textbf{ if } B \textbf{ then } S \textbf{ else } T \textbf{ endif } \{Q\}$}
\end{prooftree}


\item[Sequencing:]

\begin{prooftree}
	\AxiomC{$ \{P\} S \{Q\} $}
	\AxiomC{$ \{Q\} T \{R\} $}
	\BinaryInfC{$ \{P\} S;T \{R\} $}
\end{prooftree}

\item[While Loops:]


\begin{prooftree}
	\AxiomC{$ \{P \land B \} S \{P\} $}
	\UnaryInfC{$ \{P\} \textbf{ while } B \textbf{ do } S \textbf{ done } \{ \lnot B \land P\} $}
\end{prooftree}



\end{description*} 


\subsection{A Note on Completeness}
Due to the fact that a language (with some appropriate construct such as iteration) can create a program which will potentially never terminate, we need to prove termination conditions separately (as is done in the \textit{Probabilistic Hoare-Style Logic} paper\cite{hartogs2002prob}). In general, this is complicated.



\section{Probabilistic Hoare Logic}
When moving to probabilistic Hoare logic, we need to be able to encode some idea of probability into our Hoare triples, such as $\{ Pr(x=1) > \frac{1}{2} \} c \{ Pr(z=3) = \frac{2}{3} \}$. In \cite{rand2015vphl}, the authors propose adding one more rule to the Hoare logic rules:


\begin{description*}
\item[Coin toss:]


\begin{prooftree}
	\AxiomC{$ y \text{ free in } P $}
	\UnaryInfC{$ \{P\} y:=\texttt{toss}(p) \{ P \lhd ^y _p \} $}
\end{prooftree}

\textit{(where $P \lhd ^y _p$ is read ``P conditioned on y with probability p'')}

\end{description*} 


The difficulty with probabilistic Hoare logic is almost entire tied up in the \textit{``if''} and \textit{``while''} rules, particularly when the guard is probabilistic. Quite frankly, I don't understand these sections yet.

The paper \cite{hartogs2002prob} introduces a new syntax ``$s \oplus _\rho s'$'' which says that $s$ will occur with probability $\rho$, and $s'$ will occur with probability $1 - \rho$. This introduces a different rule than above, but it retains a similar feel:

\begin{description*}

\item[Probabilistic:]

\begin{prooftree}
	\AxiomC{$ \{ P \} C \{ R \} $}
	\AxiomC{$ \{ P \} C' \{ R' \} $}
	\BinaryInfC{$ \{P\} C \oplus _\rho C' \{ R \oplus _\rho R' \} $}
\end{prooftree}


\end{description*} 


\subsection{Example}
Using deductive logic, we can prove the following simple program which contains some probabilistic element:

\[
\{ Pr(x=1) = 1 \} (x := x+1)\oplus _{\frac{1}{2}} (x = x+2) \{ Pr(x=2) = \frac{1}{2} \land Pr(x = 3) = \frac{1}{2} \}
\]

And the complete derivation:

\vspace{-1.5em}

\begin{prooftree}
\AxiomC{ }
\UnaryInfC{$ \{ x+1=2 \} x:=x+1 \{ x=2\} $}
\UnaryInfC{$ \{ x=1 \} x:=x+1 \{ x=2 \} $}
			\AxiomC{ }
			\UnaryInfC{$ \{ x+2=3 \} x:=x+2 \{ x=3\} $}
			\UnaryInfC{$ \{ x=1 \} x:=x+2 \{ x=3\} $}
\BinaryInfC{$ \{ x=1 \} (x := x+1)\oplus _{\frac{1}{2}} (x = x+2) \{ x=2 \oplus _{\frac{1}{2}} x = 3 $}
\UnaryInfC{$ \{ x=1 \} (x := x+1)\oplus _{\frac{1}{2}} (x = x+2) \{ Pr(x=2) = \frac{1}{2} \land Pr(x = 3) = \frac{1}{2} $}
\end{prooftree}


\bibliography{citations}
\end{document}
